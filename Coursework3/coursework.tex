% Тут используется класс, установленный на сервере Papeeria. На случай, если
% текст понадобится редактировать где-то в другом месте, рядом лежит файл matmex-diploma-custom.cls
% который в момент своего создания был идентичен классу, установленному на сервере.
% Для того, чтобы им воспользоваться, замените matmex-diploma на matmex-diploma-custom
% Если вы работаете исключительно в Papeeria то мы настоятельно рекомендуем пользоваться
% классом matmex-diploma, поскольку он будет автоматически обновляться по мере внесения корректив
%

% По умолчанию используется шрифт 14 размера. Если нужен 12-й шрифт, уберите опцию [14pt]
\documentclass[14pt]{matmex}
\usepackage{enumitem}

%\documentclass[14pt]{matmex-diploma-custom}

\begin{document}
\filltitle{ru}{
    chair              = {Кафедра Системного программирования},
    title              = {Yacc Constructor Quick Graph Query},
    type               = {coursework},
    position           = {студента},
    group              = 344,
    author             = {Свитков Сергей Андреевич},
    supervisorPosition = {старший преподаватель},
    supervisor         = {Григорьев С.\,В.},
}
\maketitle
\tableofcontents



\setmonofont[Mapping=tex-text]{CMU Typewriter Text}
\bibliographystyle{ugost2008ls}
\bibliography{coursework.bib}
\end{document}
