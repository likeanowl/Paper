\documentclass[12pt]{report}
\usepackage[a4paper,left=2cm,width=16cm,right=2cm,top=3cm]{geometry}
\usepackage{float}
\usepackage{fontspec}
\usepackage{hyperref}
\usepackage{xcolor}
\usepackage{ upgreek }

\usepackage[colorinlistoftodos,prependcaption,textsize=tiny]{todonotes}

\usepackage{polyglossia}
\newfontfamily{\cyrillicfonttt}{Liberation Mono}

\usepackage{lipsum}
\usepackage{color}
\definecolor{light-gray}{gray}{0.95}

\usepackage{listings} 
\lstset{numbers=right, 
                numberstyle=\tiny,
                breaklines=true,
                backgroundcolor=\color{light-gray},
                xleftmargin=\parindent,
                numbersep=5pt} 
\lstset{language=SQL}
\lstset{basicstyle=\footnotesize\ttfamily,breaklines=true}

\usepackage{amsmath, amsthm, amssymb, amsfonts}
\usepackage{underscore}
\usepackage{textcomp}
\setdefaultlanguage{russian}
\setmainfont[Mapping=tex-text]{CMU Serif}
\newtheorem*{maindef}{Def}

\title{Мультиагентные системы. Записки}
\author{Свитков Сергей}

\begin{document}
    \maketitle
    
    \chapter *{Предисловие}
        Данный документ представляет собой записи, касающиеся изучения мультиагентных систем.
        
        Это делается исключительно в личных целях, поэтому за содержание, правильность и корректность этого текста я не несу никакой ответственности. Можно считать это потоком мыслей аутирующего третьекурсника.
        
        Следует отметить, что на момент начала написания этого документа\footnote{В дальнейшем - просто записочки} основным источником информации\footnote{Вдохновения, озарения, помутнения рассудка} является книжка Developing Multi-Agent Systems with JADE\cite{jade}.
        
        Русского текста данной книжки в настоящее время, насколько мне известно, не существует, поэтому текст записочек можно считать вольным переводом некоторых фрагментов.
    
    
    
    \tableofcontents
    
    
        
    \chapter {Об агентах и агентно-ориентированном программировании}
    \section {Определение АОП и агентов}    
        \begin{maindef}
            Агентно-Ориентированное программирование (АОП) --- довольно новая парадигма программирования, основная идея которой заключается в применении теории искуственного интеллекта к распределенным системам.
        \end{maindef}
        АОП позволяет представить программу как коллекцию сущностей, называемых Агентами.
        \begin{maindef}
            Агент --- основная сущность для АОП. Единого определения агента до сих пор нет, но все определения сходятся в том, что Агентом в АОП является автономная сущность, которая действует подобно человеку, работающему с некоторыми клиентами над своей задачей. Основные свойства агента:
            \begin{enumerate}
                \item Автономность --- Агент действует самостоятельно, без вмешательства людей, имеет контроль над своим внутренним состоянием и действиями
                \item Социальность --- Агенты взаимодействуют между собой
                \item Реактивность --- Агент реагирует на изменения среды, в которой он работает
                \item Проактивность --- Агент не изменяется при каждом изменении среды, окружающей его, но может проявлять инициативу с целью достижения своих целей.
            \end{enumerate}
            
            Дополнительные свойства:
            \begin{enumerate}
                \item Мобильность --- Агент может перемещаться по узлам сети
                \item Рациональность --- Способность агента действовать направленно для достижения своих целей
                \item Доверительность -- Агенту может быть оказано доверие, об этом позднее
                \item Benevolent --- Агент может проявлять благожелательность, всегда пытаясь выполнить запросы других агентов.
            \end{enumerate}
        \end{maindef}
        
        \section {Архитектуры}
            Архитектура мультиагентной системы позволяет поддерживать эффективное поведение в открытых системах, системах реального времени.
            
            Основные архитектуры:
            \begin{enumerate}
                \item BDI --- belief, desire, intention. 
                \item Logic-based
                \item Layered
                \item Реактивные 
            \end{enumerate}
        
\bibliographystyle{ugost2008ls}
\bibliography{Notes.bib}
        
\end{document}
