%% Простая презентация с примером включения программного кода и
%% пошаговых спецэффектов
\documentclass{beamer}
\usepackage{fontspec}
\usepackage{minted}
\usepackage{xunicode}
\usepackage{xltxtra}
\usepackage{xecyr}
\usepackage{hyperref}
\usepackage{pgfplots}
\setbeamertemplate{footline}[frame number]
\setmainfont[Mapping=tex-text]{DejaVu Serif}
\setsansfont[Mapping=tex-text]{DejaVu Sans}
\setmonofont[Mapping=tex-text]{DejaVu Sans Mono}
\usepackage{polyglossia}
\setdefaultlanguage{russian}
\usepackage{graphicx}

\begin{document}
\title{Apache Spark: что, зачем и почему?}
\author{Сергей Свитков\\{\footnotesize\textcolor{gray}{отдел разработки ПО номер 3}}}
\institute{НТЦ "Протей"}
\frame{\titlepage}

\begin{frame}\frametitle{Введение}
План доклада:
\begin{itemize}
    \item Существующие проблемы с импортом и обработкой данных с помощью dataproc
    \item Обзор Apache Spark
    \item Рассуждения о том, как решить проблемы из пункта 1
\end{itemize}
\end{frame}

\begin{frame}\frametitle{DataProc и его минусы}
Итак, казалось бы, всё замечательно:
\begin{itemize}
    \item Зачем тратить время на изучение существующих решений, если можно потратить больше времени на написание своего?
    \item Можно сделать самый кастомный в мире механизм конфигруации
    \item Да и вообще, спулеры, фетчеры и всё-всё-всё...
    \item Ну и, конечно, обработка данных на уровне базы данных! (средствами процедур баз данных, очевидно)
\end{itemize}
\end{frame}

\begin{frame}\frametitle{DataProc и его минусы}
Однако:
\begin{itemize}
    \item Один человек не может сделать продукт, по качеству и актуальности не уступающий тому, который делают 50 человек
    \item Если подобная разработка находится строго в какой-то внутренней среде, она обречена на устаревание
    \item Изучить существующую технологию проще, чем разобраться с тем, что есть сейчас
    \item Гораздо больше документации и ответов на часто возникающие вопросы
    \item При смене кого-то из кадров/срабатывании bus factor/... --- найти человека сразу же с нужными навыками невозможно
\end{itemize}
\end{frame}

\begin{frame}\frametitle{DataProc и его минусы: итоги}
Что имеем:
\begin{itemize}
    \item Огромную кодовую базу
    \item Малое количество документации
    \item Абсолютно приватные знания о проекте и о том, как им пользоваться
    \item Сложный механизм настройки
    \item Несмотря на предыдущий пункт, отсутствие гибкости инструмента
    \item Откровенную сложность работы
\end{itemize}
\end{frame}

\begin{frame}\frametitle{Что же делать?}
Варианты:
\begin{itemize}
    \item Провести координальный рефакторинг существующей кодовой базы, ядро выложить в open source
    \item Реализовать аналогичный инструмент целиком с нуля
    \item Реализовать новое решение с использованием Apache Spark
\end{itemize}
\end{frame}

\begin{frame}\frametitle{Решение с использованием Apache Spark}
Это позволит:
\begin{itemize}
    \item Унифицировать процесс импорта данных
    \item Вынести обработку данных в код
    \item Делать всё это многопоточно
    \item В случае необходимости подключить к проекту нового человека --- легко обучить его
    \item Использовать общеизвестную технологию
    \item Как следствие --- гораздо быстрее находить ответы на свои вопросы
\end{itemize}
\end{frame}

\begin{frame}\frametitle{Решение с использованием Apache Spark}
Это позволит:
\begin{itemize}
    \item Унифицировать процесс импорта данных
    \item Вынести обработку данных в код
    \item Делать всё это многопоточно
    \item В случае необходимости подключить к проекту нового человека --- легко обучить его
    \item Использовать общеизвестную технологию
    \item Как следствие --- гораздо быстрее находить ответы на свои вопросы
\end{itemize}
\end{frame}


\begin{frame}\frametitle{Apache Spark}
\begin{itemize}
    \item Фреймворк для распределенной обработки больших данных
    \item Создан в 2012 году, первый официальный релиз в 2014
    \item Open source, написан на Scala
\end{itemize}
\end{frame}

\begin{frame}\frametitle{Apache Spark}
Преимущества
\begin{itemize}
    \item Failure tolerance
    \item Абстракции данных строго типизированы
    \item Абстракции данных очень похожи на стандартные коллекции в Scala/Java
    \item Три типа абстракций: RDD, DataSet, DataFrame, подробнее о них позже
    \item Топология узлов master/slave
    \item Кэширование результатов операций
    \item Catalyst optimizer
\end{itemize}
\end{frame}

\begin{frame}\frametitle{Apache Spark}
Абстракции представления данных
\begin{itemize}
    \item RDD: 
    \begin{itemize}
        \item Строго типизированная коллекция
        \item Важен порядок операций
        \item Никак не оптимизируется автоматически
        \item Хорошо подходят для плохо структурированных данных
        \item Функциональное API
        \item Самый медленный из трёх вариантов
    \end{itemize}
\end{itemize}
\end{frame}

\begin{frame}\frametitle{Apache Spark}
Абстракции представления данных
\begin{itemize}
    \item DataSet: 
    \begin{itemize}
        \item Типизированная коллекция
        \item Функциональное API
        \item Частично оптимизируется автоматически
        \item Хорошо подходит для структурированных данных
        \item Хороший перформанс, но медленнее, чем DataFrame
    \end{itemize}
\end{itemize}
\end{frame}

\begin{frame}\frametitle{Apache Spark}
Абстракции представления данных
\begin{itemize}
    \item DataFrame: 
    \begin{itemize}
        \item Нетипизированная коллекция
        \item Взаимодействие осуществляется средствами Spark SQL
        \item Полностью оптимизируется с помощью Catalyst
        \item Подходит только для структурированных данных
        \item Самый быстрый перформанс
    \end{itemize}
\end{itemize}
\end{frame}

\begin{frame}\frametitle{Apache Spark}
Операции над данными
\begin{itemize}
    \item Transformations 
    \begin{itemize}
        \item lazy
        \item map, flatMap, filter, groupBy, ...
        \item Возвращают коллекцию
    \end{itemize}
    \item Actions
    \begin{itemize}
        \item eager
        \item reduce, fold, agregate, ...
        \item Возвращают значение (что-то, но не коллекцию)
    \end{itemize}
    \item Чтобы применить цепочку трансформаций, нужно вызвать Action
\end{itemize}
\end{frame}

\begin{frame}\frametitle{Apache Spark}
Почему не Hadoop?
\begin{itemize}
    \item Hadoop шаффлит данные после каждой операции
    \item Persistense, т.е. запись промежуточных результатов на диск
    \item Из-за этого --- низкая скорость работы
\end{itemize}
\end{frame}

\begin{frame}\frametitle{Apache Spark}
Как что-то посчитать?
\begin{itemize}
    \item Создать SparkJob
    \item Есть одна master нода и много worker нод
    \item Master нода содержит SparkContext
    \item Worker ноды содержат SparkExecutor
    \item Master нода создает коллекции с данными и распределяет их между worker нодами
    \item Они коммуницируют через cluster manager
    \item Между нодами минимизируется передача данных за счёт передачи исполяемого кода
\end{itemize}
\end{frame}

\begin{frame}\frametitle{Итоги}
Возможно, слишком амбициозное предложение, но всё же:
\begin{itemize}
    \item Использовать Apache Spark
    \item С его помощью унифицировать процессы чтения и записи данных
    \item Для каждого заказчика писать парсер его данных
    \item Запускать обработку данных не на серверах заказчика, а выделить несколько наших серверов и работать на них
    \item Это позволит сильно ускорить процесс запуска импорта данных для нового заказчика
\end{itemize}
\end{frame}

\begin{frame}\frametitle{Материалы к докладу}
\begin{itemize}
    \item https://www.coursera.org/specializations/scala
    \item https://stepik.org/course/75/syllabus
    \item https://stepik.org/course/693/syllabus
    \item https://github.com/likeanowl/Paper/blob/master/Notes/SparkBigData.md
\end{itemize}
\end{frame}

\end{document}