\documentclass[12pt]{book}
\usepackage{fontspec}
\usepackage{polyglossia}
\usepackage{amsthm}
\setdefaultlanguage{russian}
\setmainfont[Mapping=tex-text]{CMU Serif}
\newtheorem*{maindef}{Def}

\title{Мультиагентные системы. Записки}
\author{Свитков Сергей}

\begin{document}
    \maketitle
    \tableofcontents
    \section *{Предисловие}
        Данный документ представляет собой записи, касающиеся изучения мультиагентных систем.
        
        Это делается исключительно в личных целях, поэтому за содержание, правильность и корректность этого текста я не несу никакой ответственности. Можно считать это потоком мыслей аутирующего третьекурсника.
        
        Следует отметить, что на момент начала написания этого документа\footnote{В дальнейшем - просто записочки} основным источником информации\footnote{Вдохновения, озарения, помутнения рассудка} является книжка Developing Multi-Agent Systems with JADE\cite{jade}.
        
        Русского текста данной книжки в настоящее время, насколько мне известно, не существует, поэтому текст записочек можно считать вольным переводом некоторых фрагментов.
        
        
    \section *{Введение}
        \begin{maindef}
            Агентно-Ориентированное программирование (АОП) --- довольно новая парадигма программирования, основная идея которой заключается в применении теории искуственного интеллекта к распределенным системам.
        \end{maindef}
        АОП позволяет представить программу как коллекцию сущностей, называемых Агентами.
        \begin{maindef}
            Агент --- основная сущность для АОП. Его основные свойства:
            \begin{enumerate}
                \item Автономность
                \item Социальность
                \item 
            \end{enumerate}
        \end{maindef}
\bibliographystyle{ugost2008ls}
\bibliography{jade.bib}
        
\end{document}
