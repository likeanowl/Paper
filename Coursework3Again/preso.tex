%% Простая презентация с примером включения программного кода и
%% пошаговых спецэффектов
\documentclass{beamer}
\usepackage{fontspec}
\usepackage{xunicode}
\usepackage{xltxtra}
\usepackage{xecyr}
\usepackage{hyperref}
\setmainfont[Mapping=tex-text]{DejaVu Serif}
\setsansfont[Mapping=tex-text]{DejaVu Sans}
\setmonofont[Mapping=tex-text]{DejaVu Sans Mono}
\usepackage{polyglossia}
\setdefaultlanguage{russian}
\usepackage{graphicx}
\usepackage{minted}
%\lstdefinestyle{mycode}{
%  belowcaptionskip=1\baselineskip,
%  breaklines=true,
%  xleftmargin=\parindent,
%  showstringspaces=false,
%  basicstyle=\footnotesize\ttfamily,
%  keywordstyle=\bfseries,
%  commentstyle=\itshape\color{gray!40!black},
%  stringstyle=\color{red},
%  numbers=left,
%  numbersep=5pt,
%  numberstyle=\tiny\color{gray},
%}
%\lstset{escapechar=@,style=mycode}

\begin{document}
\title{Реализация библиотеки для потоковой обработки .xlsx файлов}
%%\subtitle{предварительные результаты}
\author{Свитков Сергей\\{\footnotesize\textcolor{gray}{группа 344\\научный руководитель Ю.В. Литвинов\\консультант М.В. Заведеев}}}
\institute{СПбГУ\\кафедра системного программирования}
\frame{\titlepage}

\begin{frame}\frametitle{Введение}
\begin{itemize}
    \item Веб-приложения в сфере биллинга, телекоммуникаций
    \item Различные отчеты, статистика
    \item Формат .xlsx
    \item Потоковая обработка для экономии памяти
\end{itemize}
\end{frame}

\begin{frame}\frametitle{Существующие решения}
\begin{itemize}
    \item Apache POI
    \begin{itemize}
        \item До версии 3.8 --- только работа in-memory
        \item Начиная с 3.8 --- Stream-API для записи, Event-API для чтения
        \item Часть операций всё равно только in-memory
        \item Отсутствие полных и подробных примеров работы с Stream-API
    \end{itemize}
    \item SJXLSX
    \begin{itemize}
        \item Не поддерживается с 2015 года
    \end{itemize}
    \item Excel Streaming Reader
    \begin{itemize}
        \item Разработка коммьюнити
        \item Обертка над POI
    \end{itemize}
\end{itemize}
\end{frame}

\begin{frame}\frametitle{Итоги обзора}
\begin{itemize}
    \item Stream-API POI всё равно имеет проблемы с памятью
    \item Библиотеки, реализованные коммьюнити, эти проблемы не решают
    \item Для записи в файл стоит создать свою реализацию
    \item Для чтения можно использовать парсер POI
    \item Полученную реализацию сравнить с существующими
\end{itemize}
\end{frame}

\begin{frame}\frametitle{Постановка задачи}
\begin{itemize}
    \item Реализовать библиотеку для потоковой обработки xlsx файлов
    \item Написать документацию
    \item Опубликовать библиотеку в Maven
    \item Провести апробацию
    \item Сравнить полученную реализацию с существующими
\end{itemize}
\end{frame}
%\lstset{language=java}
%\begin{frame}[fragile]\frametitle{Алгоритм}
%\begin{lstlisting}
%while (isWater()) {
%  row(boat);
%  if (crayfish) {
%    put(hand, river);
%  }
%}
%\end{lstlisting}
%\end{frame}

%\begin{frame}\frametitle{Результаты}
%\Large
%\begin{itemize}
%    \item Достигли
%    \begin{itemize}
%        \item того
%        \item сего
%    \end{itemize}
%    \item Получили
%    \item Обнаружили
%\end{itemize}
%\end{frame}
\end{document}