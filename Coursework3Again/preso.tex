%% Простая презентация с примером включения программного кода и
%% пошаговых спецэффектов
\documentclass{beamer}
\usepackage{fontspec}
\usepackage{xunicode}
\usepackage{xltxtra}
\usepackage{xecyr}
\usepackage{hyperref}
\setmainfont[Mapping=tex-text]{DejaVu Serif}
\setsansfont[Mapping=tex-text]{DejaVu Sans}
\setmonofont[Mapping=tex-text]{DejaVu Sans Mono}
\usepackage{polyglossia}
\setdefaultlanguage{russian}
\usepackage{graphicx}
\usepackage{listings}
\lstdefinestyle{mycode}{
  belowcaptionskip=1\baselineskip,
  breaklines=true,
  xleftmargin=\parindent,
  showstringspaces=false,
  basicstyle=\footnotesize\ttfamily,
  keywordstyle=\bfseries,
  commentstyle=\itshape\color{gray!40!black},
  stringstyle=\color{red},
  numbers=left,
  numbersep=5pt,
  numberstyle=\tiny\color{gray},
}
\lstset{escapechar=@,style=mycode}

\begin{document}
\title{Реализация библиотеки для потоковой обработки .xlxs файлов}
%%\subtitle{предварительные результаты}
\author{Свитков Сергей\\{\footnotesize\textcolor{gray}{группа 344\\научный руководитель Ю.В. Литвинов\\консультант М.В. Заведеев}}}
\institute{СПБГУ\\кафедра системного программирования}
\frame{\titlepage}

\begin{frame}\frametitle{Введение}
\begin{itemize}
    \item Веб-приложения
    \item Различные отчеты, статистика
    \item Формат .xlxs
\end{itemize}
\end{frame}

%\lstset{language=java}
%\begin{frame}[fragile]\frametitle{Алгоритм}
%\begin{lstlisting}
%while (isWater()) {
%  row(boat);
%  if (crayfish) {
%    put(hand, river);
%  }
%}
%\end{lstlisting}
%\end{frame}

%\begin{frame}\frametitle{Результаты}
%\Large
%\begin{itemize}
%    \item Достигли
%    \begin{itemize}
%        \item того
%        \item сего
%    \end{itemize}
%    \item Получили
%    \item Обнаружили
%\end{itemize}
%\end{frame}
\end{document}